\documentclass[11pt, a4paper,uplatex]{jsarticle}

\usepackage[dvipdfmx]{graphicx,color}
\usepackage{amsmath}
\usepackage{amssymb}
\usepackage{bm}
\usepackage{arydshln}
\usepackage{cite}
\usepackage{comment}
\usepackage{cleveref}
\usepackage{subcaption}

%\renewcommand{\figurename}{Fig. }
%\renewcommand{\tablename}{Table }

\begin{document}

\title{システム制御プロジェクト \\
自己評価書}
\author{2019年度 第5班}
\maketitle

我々の班で制作した,「人物の動画から,人物の姿と声を3Dキャラクターに置き換えてエンターテインメント性を高めた動画を出力するソフトウェア」についての評価を行う.
まず評価の前にソフトウェアの目的について整理する.
ターゲットは政治に関心がない若者でyoutubeなどのインターネット文化に馴染んでいる者とした.
こういった若者が我々のソフトウェアにより生成されたキャラクターアニメーションを見て,
愛着の湧くようなかわいらしさとキャラクター性による馴染みやすさ,内容と見た目のミスマッチやアニメ的表現・表情により面白さを感じ,
そこから少しずつ政治の用語や情勢が頭に入るようになったり,
自分から積極的に調べるようになり,若者が政治にどんどん関わるようになる,
というようなストーリーを実現できるようなソフトウェアの開発が目的である.

この目的において制作したソフトウェアの評価を行う.
現在制作したソフトウェアの持つ機能は動画から3次元の空間上のキャラクターの動きと字幕を生成するところまでであり,ボイスチェンジャーについても統合が可能な段階まで完成している.
制作物の具体的な動作については同じく提出した動画及び仕様書を参照.
動画からわかるように,動画内に写っている体のパーツについては上手くキャラクターに反映できており,
字幕もかなりの精度で再現できていることがわかる.
また,動き自体も滑らかに再現できており,キャラクターとしての馴染みやすさは表現できていて,
内容とのミスマッチ感の面白さも感じられる.
一方で,表情・エフェクト(集中線や字幕の大きさ・色・フォントの変更)が現在つけられていないのでアニメ的な面白さについては十分に表現できていない.
また,音声についてはそのまま用いることはできないので,どのような音声をつけると面白く馴染みやすいのか検証する必要がある.

\end{document}